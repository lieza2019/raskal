\documentclass[12pt]{article}

\usepackage{scicite}

\usepackage{times}

\usepackage{amsmath, amssymb}
\usepackage[dvipdfmx]{graphicx}
\usepackage{tabularx, moreverb}
\usepackage{mathrsfs}
\usepackage{theorem}
\usepackage{comment}

\newtheorem{Definition}{Definition}[section]
\newtheorem{Property}{Property}[section]
\newtheorem{Corollary}{Corollary}[section]
\newtheorem{Lemma}{Lemma}[section]
\newtheorem{Claim}{Claim}[section]
\newtheorem{Proof}{Proof.}

\renewcommand{\theProof}{}
\usepackage{lscape}
\usepackage{bussproofs}

\DeclareMathAlphabet{\mathpzc}{OT1}{pzc}{m}{it}

\topmargin 0.0cm
\oddsidemargin 0.2cm
\textwidth 16cm 
\textheight 21cm
\footskip 1.0cm

\newenvironment{sciabstract}{
\begin{quote} \bf}
{\end{quote}}

\renewcommand\refname{References and Notes}


\newcounter{lastnote}
\newenvironment{scilastnote}{
\setcounter{lastnote}{\value{enumiv}}
\addtocounter{lastnote}{+1}
\begin{list}
{\arabic{lastnote}.}
{\setlength{\leftmargin}{.22in}}
{\setlength{\labelsep}{.5em}}}
{\end{list}}


\title{Raskal, the PASCAL like rational typed procedural language}

\author
{Nobody,$^{1\ast}$Somebody,$^{1}$Anybody$^{1}$\\
\\
\normalsize{$^{1}$SOMETHING CONSTRUCTIVE  CO., LTD. Tokyo Japan}\\
\normalsize{$^\ast$To whom correspondence should be addressed;
  E-mail: noname@nowhere.com.}
}

\date{}


%%%%%%%%%%%%%%%%% END OF PREAMBLE %%%%%%%%%%%%%%%%



\begin{document}
\maketitle

\begin{displaymath}
  \begin{array}{c}
    \begin{array}{rl}
      \dfrac{
        \dfrac{}{
          \raisebox{0.6ex}[2.3ex][0ex]{
            $n \!\in\! \mathbb{I}$
          }
        }
      }{
        \{x : \text{\it Integer} \} \vdash \mathsf{var}\ x :: \mathsf{Integer} := n
      }
      & \dfrac{
          \dfrac{}{
            \raisebox{0.6ex}[2.3ex][0ex]{
              $n \!\in\! \mathbb{I}$
            }
          }
        }{
          \{x : \text{\it Integer} \} \vdash \mathsf{var}\ x := n
        }
    \end{array}  \\
    \\
    
    \begin{array}{rl}
      \dfrac{
        \dfrac{}{
          \raisebox{0.6ex}[2.3ex][0ex]{
            $r \!\in\! \mathbb{R}$
          }
        }
      }{
        \{x : \text{\it Real} \} \vdash \mathsf{var}\ x :: \mathsf{Real} := r
      }
      & \dfrac{
          \dfrac{}{
            \raisebox{0.6ex}[2.3ex][0ex]{
              $r \!\in\! \mathbb{R}$
            }
          }
        }{
          \{x : \text{\it Real} \} \vdash \mathsf{var}\ x := r
        }
    \end{array}  \\
    \\
    
    \begin{array}{rl}
      \dfrac{
        \dfrac{}{
          \raisebox{0.6ex}[2.3ex][0ex]{
            $s \!\in\! \mathbb{S}$
          }
        }
      }{
        \{x : \text{\it String} \} \vdash \mathsf{var}\ x :: \mathsf{String} := s
      }
      & \dfrac{
          \dfrac{}{
            \raisebox{0.6ex}[2.3ex][0ex]{
              $s \!\in\! \mathbb{S}$
            }
          }
        }{
          \{x : \text{\it String} \} \vdash \mathsf{var}\ x := s
        }
    \end{array}  \\
    \\
    
    \begin{array}{rl}
      \dfrac{
        \dfrac{}{
          \raisebox{0.6ex}[2.3ex][0ex]{
            $c \!\in\! \mathbb{C}$
          }
        }
      }{
        \{x : \text{\it Char} \} \vdash \mathsf{var}\ x :: \mathsf{Char} := c
      }
      & \dfrac{
          \dfrac{}{
            \raisebox{0.6ex}[2.3ex][0ex]{
              $c \!\in\! \mathbb{C}$
            }
          }
        }{
          \{x : \text{\it Char} \} \vdash \mathsf{var}\ x := c
        }
    \end{array}  \\
    \\

    \begin{array}{c}
      \dfrac{}{
        \{x : \text{\it Bot} \} \vdash \mathsf{var}\ x
      }
    \end{array}
  \end{array}
\end{displaymath}

\end{document}

